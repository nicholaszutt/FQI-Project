\section{Devices}

\begin{frame}{Device connections}
Devices, with connections, used in our research:
\vspace{0.5cm}

	\begin{columns}[T]
		\column{0.33\textwidth}
		\centering \textbf{Burlington}
		\begin{figure}[h] \centering
			\includegraphics[width=\textwidth]{images/connection_diagram_burlington.png}
			\label{fig:burlington_connections}
		\end{figure}
		
		\column{0.33\textwidth}
		\centering \textbf{Melbourne}
		\begin{figure}[h] \centering
			\includegraphics[width=\textwidth]{images/connection_diagram_melbourne.png}
			\caption*{\tiny Mainly qubits 0, 1, 2 and 14 were used.}
			\label{fig:melbourne_connections}
		\end{figure}
		
		\column{0.33\textwidth}
		\centering \textbf{Yorktown}
		\begin{figure}[h] \centering
			\includegraphics[width=\textwidth]{images/connection_diagram_ibmqx2.png}
			\label{fig:ibmqx2_connections}
		\end{figure}
		
	\end{columns}
Device connections are significant for number of gates required to run a circuit. 
\end{frame}

\begin{frame}{Example: Yorktown versus Burlington}
Yorktown and Burlington both 5 qubit devices, but give different number of required gates.

\begin{figure}[h] \centering
	\includegraphics[width=0.5\textwidth]{images/purification_ibmqx2.png}
	\caption*{\tiny Yorktown (7 CNOT gates)}
	\label{fig:pure_york}
\end{figure}

\begin{figure}[h] \centering
	\includegraphics[width=0.8\textwidth]{images/purification_burlington.png}
	\caption*{\tiny Burlington (13 CNOT gates)}
	\label{fig:pure_burl}
\end{figure}
\end{frame}

\begin{frame}{Error percentage}
Average percent error on single qubit $U_2$ and CNOT gates.
\vspace{0.5cm}
\begin{table} \centering
	\begin{tabular}{lrrrr} \toprule Backend & $U_2 (\%)$ & $CNOT (\%)$ \\ \midrule
		Burlington & 0.050 & 1.213 \\ Melbourne & 1.258 & 2.156 \\ Yorktown & 0.689 &
		2.275 \\ \bottomrule
	\end{tabular}
	\label{tb:average_errors}
\end{table}
\vspace{0.5cm}
Burlington has lowest average error, but this doesn't necessarily mean better fidelity, as the number of gates for each circuit varies per device.
\end{frame}


\section{Quantum Circuits}

\begin{frame}{The Teleportation protocol}
	
	\begin{block}{Circuit run}
		\begin{itemize}
			\item Initialize random state on first qubit.
			\item Initialize $\ket{\Phi^+}$ Bell state on bottom qubits.
			\item CNOT gate on qubit 0 and 1.
			\item Hadamard gate on qubit 0.
		\end{itemize}
	\end{block}
	
	\begin{figure}[h] \centering
		\includegraphics[width=\textwidth]{images/teleport_circuit.png}
		\label{fig:tele_circ}
	\end{figure}

\end{frame}

\begin{frame}{The Teleportation protocol}
	No operations possible after measurement on IBM Q devices. So, resort to post measurement techniques. If measurement of qubit 0 and 1 gives -1, Pauli-Z  and Pauli-X matrix are applied, respectively.
	\vspace{0.5cm}
	\begin{figure}[h] \centering
		\includegraphics[width=\textwidth]{images/Teleport_general.png}
		\label{fig:tele_gen}
	\end{figure}
	
\end{frame}

\begin{frame}{Entanglement swap}
	
	\begin{block}{Circuit run}
		\begin{itemize}
			\item Initialize random Bell-like state on top qubits.
			\item Initialize $\ket{\Phi^+}$ Bell state on bottom qubits.
			\item CNOT gate on qubit 1 and 2.
			\item Hadamard gate on qubit 1.
		\end{itemize}
	\end{block}
	
	\begin{figure}[h] \centering
		\includegraphics[width=0.8\textwidth]{images/swap_circuit.png}
		\label{fig:swap_circ}
	\end{figure}
	
\end{frame}

\begin{frame}{Entanglement swap}
	Again, resort to post measurement techniques. If measurement of qubit 1 and 2 gives -1, Pauli-Z  and Pauli-X matrix are applied, respectively. This makes an entanglement between qubits 0 and 3 of the random Bell-like state.
	\vspace{0.5cm}
		\begin{figure}[h] \centering
		\includegraphics[width=0.8\textwidth]{images/swap_general.png}
		\label{fig:swap_gen}
	\end{figure}
	
\end{frame}

\begin{frame}{Entanglement purification}
	
	\begin{block}{Circuit run}
		\begin{itemize}
			\item Initialize two Bell-like state on top and bottom qubits by rotating $\theta = \arcsin{\left(2F-1\right)}$. Where $F$ is the input fidelity.
			\item CNOT gate on qubit 1 and 3.
			\item CNOT gate on qubit 2 and 4.
		\end{itemize}
	\end{block}
	
	\begin{figure}[h] \centering
		\includegraphics[width=0.6\textwidth]{images/purification_circuit.png}
		\label{fig:puri_circ}
	\end{figure}
	
\end{frame}

\begin{frame}{Entanglement purification}
	
	\begin{block}{Circuit run (with states)}
		\begin{itemize}
			\item $\ket{\Psi} = 0.933\ket{0000}+0.067\ket{1111}
			+0.250\left(\ket{0011}+\ket{1100}\right)$
			\item $\ket{\Psi} = 0.933\ket{0000}+0.067\ket{1011}
			+0.250\left(\ket{0111}+\ket{1100}\right)$
			\item $\ket{\Psi} =
			0.933\ket{0000}+0.067\ket{0011} +0.250\left(\ket{1111}+\ket{1100}\right)$
		\end{itemize}
	\end{block}

Measuring bottom two qubits in $\ket{11}$ gives the top two qubit state
$\ket{\Psi_{1,2}} = \frac{1}{\sqrt{2}}\left(\ket{11} +\ket{00}\right) =
\ket{\Phi^+}$ with a probability of $2\cdot0.250^2 = 0.125$.
	
	\begin{figure}[h] \centering
		\includegraphics[width=0.5\textwidth]{images/purification_circuit.png}
		\label{fig:puri_circ}
	\end{figure}
	
\end{frame}

\begin{frame}{Grover's search algorithm}
		
	\begin{block}{Circuit run}
		\begin{itemize}
			\item $\ket{\Psi} = 0.933\ket{0000}+0.067\ket{1111}
			+0.250\left(\ket{0011}+\ket{1100}\right)$
			\item $\ket{\Psi} = 0.933\ket{0000}+0.067\ket{1011}
			+0.250\left(\ket{0111}+\ket{1100}\right)$
			\item $\ket{\Psi} =
			0.933\ket{0000}+0.067\ket{0011} +0.250\left(\ket{1111}+\ket{1100}\right)$
		\end{itemize}
	\end{block}
	

	\begin{figure}[h] \centering
		\includegraphics[width=\textwidth]{images/grover_circuit.png}
		\label{fig:grov_circ}
	\end{figure}
	
\end{frame}

%%% Local Variables:
%%% mode: latex
%%% TeX-master: "presentation"
%%% End: