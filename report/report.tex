% Journal Article
% LaTeX Template
%----------------------------------------------------------------------------------------
%	PACKAGES AND OTHER DOCUMENT CONFIGURATIONS
%----------------------------------------------------------------------------------------

\documentclass[twoside,twocolumn]{article}

\usepackage{blindtext} % Package to generate dummy text throughout this template 

\usepackage[sc]{mathpazo} % Use the Palatino font
\usepackage[T1]{fontenc} % Use 8-bit encoding that has 256 glyphs
\linespread{1.05} % Line spacing - Palatino needs more space between lines
\usepackage{microtype} % Slightly tweak font spacing for aesthetics

\usepackage[english]{babel} % Language hyphenation and typographical rules
\usepackage{graphicx}
\usepackage{amsmath}
\usepackage{amssymb}

\usepackage[hmarginratio=1:1,top=32mm,columnsep=20pt]{geometry} % Document margins
\usepackage[format=plain, small,labelfont=bf,up,textfont=it,up]{caption} 
\usepackage{booktabs} % Horizontal rules in tables

\usepackage{lettrine} % The lettrine is the first enlarged letter at the beginning of the text
\usepackage{units}

\usepackage{enumitem} % Customized lists
\setlist[itemize]{noitemsep} % Make itemize lists more compact

\usepackage{abstract} % Allows abstract customization
\renewcommand{\abstractname}{\vspace{-\baselineskip}}
\renewcommand{\abstractnamefont}{\normalfont\bfseries} % Set the "Abstract" text to bold
% Set the abstract itself to small italic text
\renewcommand{\abstracttextfont}{\normalfont\small\itshape} 

\usepackage{titlesec} % Allows customization of titles
\renewcommand\thesection{\Roman{section}} % Roman numerals for the sections
\renewcommand\thesubsection{\roman{subsection}} % roman numerals for subsections
% Change the look of the section titles
\titleformat{\section}[block]{\large\scshape\centering}{\thesection.}{1em}{}
% Change the look of the section titles
\titleformat{\subsection}[block]{\large}{\thesubsection.}{1em}{}

\usepackage{fancyhdr} % Headers and footers
\pagestyle{fancy} % All pages have headers and footers
\fancyhead{} % Blank out the default header
\fancyfoot{} % Blank out the default footer
% Custom header text
\fancyhead[C]{Argon Simulation $\bullet$ March 2020} % $\bullet$ Vol. XXI, No. 1}
\fancyfoot[RO,LE]{\thepage} % Custom footer text

\usepackage{titling} % Customizing the title section
\usepackage{hyperref} % For hyperlinks in the PDF

% SI Units like the fancy Angstrom
\usepackage{siunitx}

% Using BibTex, with specified choices of what to include in bibliography
\usepackage[backend=bibtex,
sorting=none, % sort entries by appearance
isbn=true,
doi=true,
url=false,
eprint=false,
abbreviate=false,
style=numeric]{biblatex}

% adding the bibliography file
\addbibresource{references}


% ----------------------------------------------------------------------------------------
%	TITLE SECTION
%----------------------------------------------------------------------------------------

\setlength{\droptitle}{-4\baselineskip} % Move the title up

\title{\textbf{Quantum State Tomography and Post Measurement Analysis in Qiskit}} % Article title
\author{%
\textsc{Jerry Kamer, Roel van Silfhout, Nicholas Zutt}
\\[1ex]
\normalsize{Delft University of Technology}\\ % Your institution
}
\date{April 24, 2020} % Leave empty to omit a date
\renewcommand{\maketitlehookd}{%
\begin{abstract}
  \noindent The IBM Quantum Experience is a public platform for executing
quantum circuits on superconducting back-ends. We execute the Teleportation
protocol, Entanglement Swapping, Entanglement Purification and Grover's
Algorithm on three superconducting devices available from IBMQ. We present the
results using the appropriate post-measurement selection techniques and
reconstruct the output states through state tomography. We analyze the
performance of the three devices and put forth an explanation for the
discrepancies in the results between the three devices.
\end{abstract}


%%% Local Variables:
%%% mode: latex
%%% TeX-master: "report"
%%% End:
}

%----------------------------------------------------------------------------------------

\begin{document}
	\maketitle

% ----------------------------------------------------------------------------------------
%	ARTICLE CONTENTS
%----------------------------------------------------------------------------------------

\section{Introduction}

\lettrine[nindent=0em,lines=3]{C} omputer simulations have arisen as powerful
tools for investigating the molecular dynamics of systems composed of
many particles. This report simulates the noble gas Argon at various temperatures
and densities, probing the three states of matter, and investigates a set of
macroscopic observables, the specific heat, pressure, pair correlation function
and diffusion constants of the system. \cite{nielsen10_quant}

In order to simulate a system of particles accurately, the first step is to
choose an appropriate potential with which to model the inter-atomic forces
exchanged between the particles in motion. We use a mathematically simple model
that has been popular historically due to its computational efficiency and which
has the added benefit of being especially accurate for noble gases. It assumes
dipole-dipole interaction between neutral atoms, and includes a repulsive term
for short distances. This is the Lennard-Jones potential, which has the form


%%% Local Variables:
%%% mode: latex
%%% TeX-master: "report"
%%% End:




\section{Conclusion}

Our research project was mainly focused on the fidelities of different circuits and if we were able to improve those for readout error using state tomography on various circuits. We conclude that we succeeded in recreating the final states using the post measurement method (except for the Grover circuit), since the simulator fidelities for the four circuits are approximately 1. The noisy simulator overestimated the device results for each circuit, and the error correction improves the fidelity for all circuits.  
%%% Local Variables:
%%% mode: latex
%%% TeX-master: "report"
%%% End:

%	REFERENCE LIST
\printbibliography

% ----------------------------------------------------------------------------------------
\end{document}