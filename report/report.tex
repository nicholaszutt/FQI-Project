% Journal Article
% LaTeX Template
%-----------------------------------------------------------------------------
%	PACKAGES AND OTHER DOCUMENT CONFIGURATIONS
%-----------------------------------------------------------------------------

\documentclass[twoside,twocolumn]{article}

\usepackage{blindtext} % Package to generate dummy text throughout this template 

\usepackage[sc]{mathpazo} % Use the Palatino font
\usepackage[T1]{fontenc} % Use 8-bit encoding that has 256 glyphs
\linespread{1.05} % Line spacing - Palatino needs more space between lines
\usepackage{microtype} % Slightly tweak font spacing for aesthetics

\usepackage[english]{babel} % Language hyphenation and typographical rules
\usepackage{graphicx}
\usepackage{amsmath}
\usepackage{amssymb}
\DeclareMathOperator{\Tr}{Tr}
\usepackage[hmarginratio=1:1,top=32mm,columnsep=20pt]{geometry} % Document margins
\usepackage[format=plain, small,labelfont=bf,up,textfont=it,up]{caption} 
\usepackage{booktabs} % Horizontal rules in tables
\usepackage[format=plain, labelfont=bf,textfont=it]{subcaption}

\usepackage{lettrine} % The lettrine is the first enlarged letter at the beginning of the text
\usepackage{units}

\usepackage{xcolor} % adding color to words

\usepackage{braket} % braket notation

\usepackage{nameref} %to reference name of a section
\usepackage{hyperref}

\usepackage{enumitem} % Customized lists
\setlist[itemize]{noitemsep} % Make itemize lists more compact

\usepackage{abstract} % Allows abstract customization
\renewcommand{\abstractname}{\vspace{-\baselineskip}}
\renewcommand{\abstractnamefont}{\normalfont\bfseries} % Set the "Abstract" text to bold
% Set the abstract itself to small italic text
\renewcommand{\abstracttextfont}{\normalfont\small\itshape} 

\usepackage{titlesec} % Allows customization of titles
\renewcommand\thesection{\Roman{section}} % Roman numerals for the sections
\renewcommand\thesubsection{\roman{subsection}} % roman numerals for subsections
% Change the look of the section titles
\titleformat{\section}[block]{\large\scshape\centering}{\thesection.}{1em}{}
% Change the look of the section titles
\titleformat{\subsection}[block]{\large}{\thesubsection.}{1em}{}

\usepackage{fancyhdr} % Headers and footers
\pagestyle{fancy} % All pages have headers and footers
\fancyhead{} % Blank out the default header
\fancyfoot{} % Blank out the default footer
% Custom header text
\fancyhead[C]{Qiskit $\bullet$ April 2020} % $\bullet$ Vol. XXI, No. 1}
\fancyfoot[RO,LE]{\thepage} % Custom footer text

\usepackage{titling} % Customizing the title section
\usepackage{hyperref} % For hyperlinks in the PDF
\usepackage[page]{appendix}
\usepackage{pdfpages}
\renewcommand{\appendixpagename}{\vspace*{\fill}\centering Transpiled Circuits
\vspace*{\fill}}

% SI Units like the fancy Angstrom
\usepackage{siunitx}
\usepackage{bbold}

% Using BibTex, with specified choices of what to include in bibliography
\usepackage[backend=bibtex,
sorting=none, % sort entries by appearance
isbn=true,
doi=false,
url=true,
eprint=false,
abbreviate=false,
style=numeric]{biblatex}

% adding the bibliography file
\addbibresource{references}


%-----------------------------------------------------------------------------
%	TITLE SECTION
%-----------------------------------------------------------------------------

\setlength{\droptitle}{-4\baselineskip} % Move the title up

\title{\textbf{Quantum State Tomography and Post Measurement Analysis in Qiskit}} % Article title
\author{%
\textsc{Jerry Kamer, Roel van Silfhout, Nicholas Zutt}
\\[1ex]
\normalsize{Delft University of Technology}\\ % Your institution
}
\date{April 24, 2020} % Leave empty to omit a date
\renewcommand{\maketitlehookd}{%
\begin{abstract}
  \noindent The IBM Quantum Experience is a public platform for executing
  quantum circuits on superconducting back-ends. We execute the Teleportation
  protocol, Grover's search algorithm, Entanglement Swapping and Entanglement
  Purification on three superconducting devices available from IBMQ. We analyze
  the results and DO MORE COOL STUFF.
\end{abstract}


%%% Local Variables:
%%% mode: latex
%%% TeX-master: "report"
%%% End:
}

%-----------------------------------------------------------------------------

\begin{document}
	\maketitle

%-----------------------------------------------------------------------------
%	ARTICLE CONTENT
%-----------------------------------------------------------------------------

\section{Introduction}

\lettrine[nindent=0em,lines=3]{Q} uantum computers exploit quantum mechanical
phenomena in order to perform calculations and manipulate data in ways that
would be impossible on classical computers. Quantum algorithms have been
developed to teleport data from one place to another, search databases
efficiently, and factor large numbers quickly, to name a few possible applications. These algorithms will comprise
the building blocks of the quantum computers of the future, and the current
efforts towards realizing a truly universal quantum computer centre around
improving the manipulation of quantum bits (or qubits), the basic units of
computation in these algorithms.

There are many approaches to creating these qubits. Useful qubits instantiate a
set of properties that can be at odds with each other at times. For example, a
desirable trait like the ability to accurately control the state of the qubit
often conflicts with the desire for the qubit to be long-lived (i.e. to have a
long coherence time where its quantum state is safe from environmental
degradation).

The various implementations make the trade-off between desirable traits in
different ways, and a leading approach for creating qubits uses LC circuits in
superconductors \cite{kjaergaard19_super_qubit}. Already in 2014, the first
demonstrated universal gate set on superconducting qubits with an average gate
fidelity over 99 per cent for all gates was realized
\cite{barends14_super_quant_circuit_at_surfac}. As the field has progressed
since then, superconducting qubits have only become more attractive as the
building blocks for quantum computers. Now, multi-qubit devices that use
superconductors are available publicly, and capable of performing quantum
computations that implement small (less than 16 qubits, with limited depth)
circuits.

We have simulated and executed a handful of foundational circuits on
superconducting devices provided publicly by the IBM Quantum Experience. We use
Qiskit, an open-source quantum computing software development
platform which has quickly become the most popular means with which to program
circuits on publicly available quantum computers. Using
1- and 2-qubit state tomography and post-measurement selection schemes, we
reconstruct the average output to characterize the fidelity with
which three different superconductor devices implement our chosen circuits.

In this report we present the results of these computations in order to
discuss the current state of development for publicly available superconducting
quantum computers. Our aim is to characterize the fidelity with which our
circuits can be implemented on the devices, and as such it is necessary to 

The purpose of this report is to examine various circuit in order to determine
if the post measurement method is valid for different circuits and if we can
improve the results form various devices by implementing 1- and 2 qubit state
tomography. In the case of for example quantum teleportation or entanglement
swapping circuit it is impossible to run the original circuits on real devices
since they require that initial measurements dictate which operators are used on
the qubits. This is impossible on IBMQ backends because they do not support
operations after measurements. The goal of these measurements is to determine
the fidelity of a prepared state to a target pure state. To this end we analyze
density matrices and Pauli set plots of the final states. Our state tomography
results also allow us to correct for readout error, which is the error caused by
the measurement of the qubits in different basis \cite{nielsen10_quant}.

In this report we first discuss the theory needed to understand state tomography
and readout error correction. Then we present our methods for developing Qiskit
software and explain the relevant circuits. In results chapter the measurement
data is shown and from this data we derive important performance metrics like
the fidelity for each circuit on different backends. Finally in the last chapter
we discuss the conclusion, recommendations and outlook.

%%% Local Variables:
%%% mode: latex
%%% TeX-master: "report"
%%% End:


\section{Theory}
In a classical computer its internal state is measured at different points in time in order to debug the system. However, for a quantum computer, the analogy would be the measurement of its density matrix, which is called state tomography. We first define the density matrix of a single qubit:
\begin{equation}
\rho=\frac{1}{2}\left(I+\sum_i\alpha_i\sigma_i\right)
\end{equation}

\section{Devices} Superconducting qubits are composed of an inductor $L$
connected to a capacitor $C$. These are described by the equations of motion of
a harmonic oscillator, with flux $\phi$ through the inductor playing the role of
the canonical position which oscillates out of phase with the charge $Q$ on the
capacitor, this charge playing the part of the canonical momentum
\cite{devoret04_implem_qubit_with_super_integ_circuit}. This LC oscillator has a
resonance frequency that can be specifically engineered as it is given by
$\omega_0 = \nicefrac{1}{\sqrt{LC}}$. The superconducting nature of the
oscillator is of critical importance to maintaining coherence as a superposition
of the ground and first excited states of the LC oscillator decays on a time
scale given by $\nicefrac{1}{RC}$, where $R$ is the resistance. LC circuits, as
macroscopic qubits, are attractive partly because their parameters can be
engineered to whatever values best suit the system in question, but it should be
noted that their size also entail some individuality to each qubit that may lead
to variations in their functioning.

\begin{figure}[h] \centering
\includegraphics[width=0.48\textwidth]{images/energy_spacing_transmon.png}
  \caption{The energy spacing between levels of an LC oscillator are ordinarily
even, anharmonicity, introduced through the inclusion of a Josephson tunnel
junction, splits the level spacing. Figure from
\cite{dickel20_how_to_make_artif_atoms}.}
  \label{fig:energy_spacing_transmon}
\end{figure}

A simple LC circuit alone, though, cannot be an effective qubit. In order to be
able to control the qubit state effectively, the transition frequency between
the states $|0\rangle$ and $|1\rangle$ have to be different enough from all
other transitions, but of course, all transitions between neighbouring states in
the harmonic oscillator potential are the same size. A degree of anharmonicity
is brought in by a Josephson tunnel element with its own non-linear inductance
which is added to the circuit in parallel with the capacitor, make it possible
to individually drive transitions between specific levels in the LC oscillator,
as shown in Fig. \ref{fig:energy_spacing_transmon}. The system's dynamics can
then be effectively confined to just the two lowest energy levels as long as the
driving frequencies for transitions are properly tuned
\cite{devoret04_implem_qubit_with_super_integ_circuit}.

\begin{figure}[h] \centering
\includegraphics[width=0.48\textwidth]{images/transmon_diagram.png}
  \caption{A schematic of a transmon qubit. The left superconductor maintains
anharmonicity while the right superconductor helps protect it from charge noise.
Figure from \cite{dickel20_how_to_make_artif_atoms}.}
  \label{fig:transmon}
\end{figure}

The devices at the IBM Quantum Experience are transmon qubits. This type of
superconducting charge qubit, shown in Fig. \ref{fig:transmon}, consists of a
qubit circuit coupled to a cavity circuit (a harmonic LC oscillator) which as a
result has a reduced sensitivity to charge noise and has recently shown
coherence times of up to 95$\mu$s and relaxation times of around 70$\mu$s.
\cite{koch07_charg_insen_qubit_desig_deriv,
rigetti12_super_qubit_waveg_cavit_with}. Three devices at the IBM Quantum
Experience were chosen to be used in this project. In this section we will
provide an overview of these devices to provide context for the results that
will be discussed below. IBM names its superconducting devices after major
cities, and those we are interested in are code-named Burlington, Melbourne and
Yorktown.

\subsection{Burlington} When comparing the different backends, the first
parameters to take into account in their characterization is the error rates for
single- and two-qubit operations. All three backends use the same universal gate
set, which make their error rates directly comparable. As can be seen in Fig.
\ref{fig:burlington_connections}, this backend has the best gate error rates of
any backend we will consider, with a single-qubit U2 error rate less than
0.065\% for all qubits and a CNOT error rate of less than 1.645\%.

\begin{figure}[h] \centering
\includegraphics[width=0.48\textwidth]{images/connection_diagram_burlington.png}
  \caption{The T-shaped Burlington device. Though a device with relatively small
error rates, the limited connectivity play a big role in the device's
performance, as many extra gates are needed to implement circuits when there are
few direct connections. Figure from \cite{ibmq_burlington}.}
  \label{fig:burlington_connections}
\end{figure}

In the Melbourne backend (Fig. \ref{fig:melbourne_connections}), single-qubit
gate errors rise to 0.746\%, though the qubits used in all the circuits
implemented here (qubits numbered 0, 1, 2 and 14) all have errors under 0.07\%.
CNOT error between the four qubits of interest to our circuits stays below 4\%.
When choosing which qubits to use for computation, the transpiling program seems
to favour implementation on qubits with the lowest errors (hence the avoidance
of the noisy qubit 13).

The individual character of the qubits is most easily seen in the Melbourne
device. Due to variability in the manufacture of the macroscopic qubits, some
come out noisier than others, and the connections between them likewise suffer
some in-homogeneity.

\begin{figure}[h] \centering
\includegraphics[width=0.48\textwidth]{images/connection_diagram_melbourne.png}
  \caption{The largest publicly available device at IBM Q. The individual
character of the transmons is visible in this diagram through the large
variability in the error rates. Figure from \cite{ibmq_16_melbourne}.}
  \label{fig:melbourne_connections}
\end{figure}

The final backend under consideration, shown in Fig.
\ref{fig:yorktown_connections} comes out between Melbourne and Burlington with
an average single-qubit error rate of 0.05\% and average CNOT error of 2.276\%.
However, as we will see in the results, these gate error rates are far from the
whole story when it comes to explaining the performance of various backends.

\begin{figure}[h] \centering
\includegraphics[width=0.48\textwidth]{images/connection_diagram_ibmqx2.png}
  \caption{The most densely connected 5-qubit device at IBM Q. As we will see,
equally as important as the single-qubit and CNOT error rates is the degree of
connectivity in a device. Figure from \cite{ibmq_yorktown}.}
  \label{fig:yorktown_connections}
\end{figure}

One consideration that is not reflected in the gate error rates which has a
significant impact on the fidelity of the outputs of circuits is the number of
direct connections between qubits. Burlington may have the lowest average error
rates for gate operations of all the three devices (shown in Table
\ref{tb:average_errors}), but as we will show, this doesn't translate to better
output fidelities in every case. In fact, because the number of gates required
to implement each circuit will vary from device to device, the optimal backend
on which to conduct a given computation will ultimately depend more on the
configuration of each backend, rather than gate errors.

\begin{table} \centering
  \begin{tabular}{lrrrr} \toprule Backend & $U_2 (\%)$ & $CNOT (\%)$ \\ \midrule
Burlington & 0.050 & 1.213 \\ Melbourne & 1.258 & 2.156 \\ Yorktown & 0.689 &
2.275 \\ \bottomrule
  \end{tabular}
  \caption{Average percent error on single-qubit $U_2$ gates and CNOT gates for
the three devices used to implement the circuits of interest. All averages taken
from data provided at \cite{ibmq_burlington,ibmq_16_melbourne,ibmq_yorktown}.}
  \label{tb:average_errors}
\end{table}

%%% Local Variables:
%%% mode: latex
%%% TeX-master: "report"
%%% End:

\section{Quantum Circuits}
For this research we have chosen to execute the Teleportation protocol, Grover's search algorithm, Entanglement Swapping and Entanglement Purification on the devices. The circuits are all made using an open source software development kit called Qiskit, which uses Python as its main programming language. The written codes, made with Jupyter Notebook, for the circuits can be found in appendix {\color{red} \emph{number}}. In the following we will describe the circuits, mainly focussing on ... and {\color{red}\emph{the measurement of the circuits}}. Thereafter, the measurement protocols will be described.

\subsection{The Teleportation protocol}
Quantum teleportation is a procedure where a quantum state can be transmitted from one location to the other. Usually this is done by creating an arbitrary state $\ket{\psi}$ on the first qubit and a $\ket{\Phi^+}$ Bell-state on the second and third qubit, as can be seen in figure \ref{fig:telgen}.
\begin{figure}[h]
	\includegraphics[width=0.48\textwidth]{images/Teleport_general.png}
	\caption{General teleportation protocol circuit. Here the $\ket{\Phi^+}$ Bell-state is denoted as $\ket{\beta_{00}}$.  \cite{nielsen10_quant}}
	\label{fig:telgen}
\end{figure}
Subsequently, a CNOT gate is applied to the second qubit and a Hadamard gate to the first qubit. Now the first and second qubit will be measured in the Z-basis, with the measurement results being $M_1$ and $M_2$, respectively. A X- and/or Z-gate is applied to the third qubit depending on the measurement outcome. If $M_1 = +1$ a Z-gate will be applied and if $M_2 = +1$ a X-gate will be applied. This will result in the state $\ket{\psi}$ being teleported to the third qubit.

However, to know if the state $\ket{\psi}$ is properly transported, the third qubit must be measured. Unfortunately, it is not possible (yet) to apply a gate after a measurement on the used devices. So, we will have to resort to post measurement techniques, as the gates cannot be applied to the third qubit. The circuit that is send to the devices is presented in figure \ref{fig:telcir}.
\begin{figure}[h]
	\includegraphics[width=0.48\textwidth]{images/teleport_circuit.png}
	\caption{Teleport circuit used for the measurements. (The operators on the first qubit were randomized every run.)}
	\label{fig:telcir}
\end{figure}
As one can easily see the gates on the third qubit are absent. To account for this, a Pauli-X or Pauli-Z matrix is applied to the final state on the third qubit, after the measurement. Which works similar to the general protocol: if $M_1 = +1$ a Pauli-Z matrix will be applied and if $M_2 = +1$ a Pauli-X matrix will be applied. This will, in post measurement, result in the state $\ket{\psi}$ being teleported to the third qubit.

\subsection{Grover's search algorithm}
The Grover search algorithm can find an input given to a black box with a high likeliness. In our case, the input that is given is a unitary 4x4 diagonal matrix, with three values being +1 and one being -1. The algorithm can find which one of the numbers on the diagonal is -1. The circuit that is used in the measurements is presented in figure \ref{fig:grocir}.
\begin{figure}[h]
	\includegraphics[width=0.48\textwidth]{images/grover_circuit.png}
	\caption{Grover's search algorithm circuit used for the measurements. (The position of the -1 value is randomized every run.)}
	\label{fig:grocir}
\end{figure}
For explanatory reasons, let us choose the unitary matrix to be:
\begin{equation*}
U = 
\begin{bmatrix}
1 & 0 & 0 & 0 \\
0 & 1 & 0 & 0 \\
0 & 0 & 1 & 0 \\
0 & 0 & 0 & -1 
\end{bmatrix}
\end{equation*} 
$U$ is randomized for every run of the circuit on a device or simulator. The two qubit will both start in the $\ket{0}$ state. After the application of the Hadamard gates the total state will be: $\ket{\Psi} = \frac{1}{2}\left(\ket{00}+\ket{01}+\ket{10}+\ket{11}\right)$. Now $U$ is applied giving: $\ket{\Psi} = \frac{1}{2}\left(\ket{00}+\ket{01}+\ket{10}-\ket{11}\right)$.
The part after the unitary in figure \ref{fig:grocir} is important for the Grover's search algorithm and does an inversion about the mean. This inverts the constants multiplied with each state around the mean of the total. In this case the mean is $\frac{3\cdot\frac{1}{2}-\frac{1}{2}}{4} = \frac{1}{4}$. Inverting $\frac{1}{2}$ about the mean, means that it will become 0. For $-\frac{1}{2}$ it will become 1, thus making $\ket{11}$ the only state left. Measuring this will result in $M_1 = M_2 = +1$. This result is related to where in $U$ the -1 value is positioned and in this case the result shows it is in the bottom right corner (where we positioned -1 in the first place).

\subsection{Entanglement swap}



\section{Results}

\begin{frame}{Results}

\end{frame}

\section{Future Outlook}

%%% Local Variables:
%%% mode: latex
%%% TeX-master: "presentation"
%%% End:


\section{Conclusion}

Our research project was mainly focused on the fidelities of different circuits and if we were able to improve those for readout error using state tomography on various circuits. We conclude that we succeeded in recreating the final states using the post measurement method (except for the Grover circuit), since the simulator fidelities for the four circuits are approximately 1. The noisy simulator overestimated the device results for each circuit, and the error correction methods for one- and two-qubit systems improve the fidelity with respect to the device. The optimal backend for executing the relevant circuits is Yorktown, because it gives the best fidelity and the Pauli sets
%%% Local Variables:
%%% mode: latex
%%% TeX-master: "report"
%%% End:

%	REFERENCE LIST
\printbibliography

\section*{Acknowledgements}
We acknowledge use of the IBM Q for this work. The views expressed are those of the authors and do not reflect the official policy or position of IBM or the IBM Q team.

\section*{GitHub}
All the code used to implement the circuits and construct the plots for this
report can be found online at:
\href{https://github.com/nicholaszutt/FQI-Project}{https://github.com/nicholaszutt/FQI-Project}.

%%% Local Variables:
%%% mode: latex
%%% TeX-master: "report"
%%% End:

\appendix
\clearpage  % move to another page
\begin{appendices}
%   The following sections include images of the transpiled circuits on the
%   different backends, followed by the code used to implement the circuits and
%   sample output.
  \clearpage % move to another page

\section*{Transpiled Circuits}
\begin{figure*}
  % \centering
  \includegraphics[width=0.6\textwidth]{images/swap_ibmqx2.png}
  \caption{Implementing swap on the Yorktown and Melbourne backends use the
    minimal number of gates.}
  \label{fig:swap_york_trans}
\end{figure*}
  
\begin{figure*}
  \centering
  \includegraphics[width=\textwidth]{images/swap_burlington.png}
  \caption{The implementation of the swapping protocol requires many more
    2-qubit gates on the Burlington device due to it T-shape and low
    inter-connectivity. This circuit should be compared to Fig.
    \ref{fig:swap_york_trans}, with just three CNOT gates.}
  \label{fig:swap_burl_trans}
\end{figure*}

\end{appendices}

%%% Local Variables:
%%% mode: latex
%%% TeX-master: "report"
%%% End:
\label{apen} % reference to the apendix

\clearpage
\includepdf[pages=-]{../Circuits/Teleportation.pdf}
\includepdf[pages=-]{../Circuits/Swap.pdf}
\includepdf[pages=-]{../Circuits/Purification.pdf}
\includepdf[pages=-]{../Circuits/Grover.pdf}

%-----------------------------------------------------------------------------
\end{document}