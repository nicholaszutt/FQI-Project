% Journal Article
% LaTeX Template
%----------------------------------------------------------------------------------------
%	PACKAGES AND OTHER DOCUMENT CONFIGURATIONS
%----------------------------------------------------------------------------------------

\documentclass[twoside,twocolumn]{article}

\usepackage{blindtext} % Package to generate dummy text throughout this template 

\usepackage[sc]{mathpazo} % Use the Palatino font
\usepackage[T1]{fontenc} % Use 8-bit encoding that has 256 glyphs
\linespread{1.05} % Line spacing - Palatino needs more space between lines
\usepackage{microtype} % Slightly tweak font spacing for aesthetics

\usepackage[english]{babel} % Language hyphenation and typographical rules
\usepackage{graphicx}
\usepackage{amsmath}
\usepackage{amssymb}

\usepackage[hmarginratio=1:1,top=32mm,columnsep=20pt]{geometry} % Document margins
\usepackage[format=plain, small,labelfont=bf,up,textfont=it,up]{caption} 
\usepackage{booktabs} % Horizontal rules in tables

\usepackage{lettrine} % The lettrine is the first enlarged letter at the beginning of the text
\usepackage{units}

\usepackage{enumitem} % Customized lists
\setlist[itemize]{noitemsep} % Make itemize lists more compact

\usepackage{abstract} % Allows abstract customization
\renewcommand{\abstractname}{\vspace{-\baselineskip}}
\renewcommand{\abstractnamefont}{\normalfont\bfseries} % Set the "Abstract" text to bold
% Set the abstract itself to small italic text
\renewcommand{\abstracttextfont}{\normalfont\small\itshape} 

\usepackage{titlesec} % Allows customization of titles
\renewcommand\thesection{\Roman{section}} % Roman numerals for the sections
\renewcommand\thesubsection{\roman{subsection}} % roman numerals for subsections
% Change the look of the section titles
\titleformat{\section}[block]{\large\scshape\centering}{\thesection.}{1em}{}
% Change the look of the section titles
\titleformat{\subsection}[block]{\large}{\thesubsection.}{1em}{}

\usepackage{fancyhdr} % Headers and footers
\pagestyle{fancy} % All pages have headers and footers
\fancyhead{} % Blank out the default header
\fancyfoot{} % Blank out the default footer
% Custom header text
\fancyhead[C]{Argon Simulation $\bullet$ March 2020} % $\bullet$ Vol. XXI, No. 1}
\fancyfoot[RO,LE]{\thepage} % Custom footer text

\usepackage{titling} % Customizing the title section
\usepackage{hyperref} % For hyperlinks in the PDF

% SI Units like the fancy Angstrom
\usepackage{siunitx}

% Using BibTex, with specified choices of what to include in bibliography
\usepackage[backend=bibtex,
sorting=none, % sort entries by appearance
isbn=true,
doi=true,
url=false,
eprint=false,
abbreviate=false,
style=numeric]{biblatex}

% adding the bibliography file
\addbibresource{references}


% ----------------------------------------------------------------------------------------
%	TITLE SECTION
%----------------------------------------------------------------------------------------

\setlength{\droptitle}{-4\baselineskip} % Move the title up

\title{\textbf{Quantum State Tomography and Post Measurement Analysis in Qiskit}} % Article title
\author{%
\textsc{Jerry Kamer, Roel van Silfhout, Nicholas Zutt}
\\[1ex]
\normalsize{Delft University of Technology}\\ % Your institution
}
\date{April 24, 2020} % Leave empty to omit a date
\renewcommand{\maketitlehookd}{%
\begin{abstract}
  \noindent The IBM Quantum Experience is a public platform for executing
  quantum circuits on superconducting back-ends. We execute the Teleportation
  protocol, Grover's search algorithm, Entanglement Swapping and Entanglement
  Purification on three superconducting devices available from IBMQ. We analyze
  the results and DO MORE COOL STUFF.
\end{abstract}


%%% Local Variables:
%%% mode: latex
%%% TeX-master: "report"
%%% End:
}

%----------------------------------------------------------------------------------------

\begin{document}
	\maketitle

% ----------------------------------------------------------------------------------------
%	ARTICLE CONTENTS
%----------------------------------------------------------------------------------------

\section{Introduction}

\lettrine[nindent=0em,lines=3]{Q} uantum computers exploit quantum mechanical
phenomena in order to perform calculations and manipulate data in ways that
would be impossible on classical computers. Quantum algorithms have been
developed to teleport data from one place to another, search databases
efficiently, and factor large numbers quickly, to name a few possible applications. These algorithms will comprise
the building blocks of the quantum computers of the future, and the current
efforts towards realizing a truly universal quantum computer centre around
improving the manipulation of quantum bits (or qubits), the basic units of
computation in these algorithms.

There are many approaches to creating these qubits. Useful qubits instantiate a
set of properties that can be at odds with each other at times. For example, a
desirable trait like the ability to accurately control the state of the qubit
often conflicts with the desire for the qubit to be long-lived (i.e. to have a
long coherence time where its quantum state is safe from environmental
degradation).

The various implementations make the trade-off between desirable traits in
different ways, and a leading approach for creating qubits uses LC circuits in
superconductors \cite{kjaergaard19_super_qubit}. Already in 2014, the first
demonstrated universal gate set on superconducting qubits with an average gate
fidelity over 99 per cent for all gates was realized
\cite{barends14_super_quant_circuit_at_surfac}. As the field has progressed
since then, superconducting qubits have only become more attractive as the
building blocks for quantum computers. Now, multi-qubit devices that use
superconductors are available publicly, and capable of performing quantum
computations that implement small (less than 16 qubits, with limited depth)
circuits.

We have simulated and executed a handful of foundational circuits on
superconducting devices provided publicly by the IBM Quantum Experience. We use
Qiskit, an open-source quantum computing software development
platform which has quickly become the most popular means with which to program
circuits on publicly available quantum computers. Using
1- and 2-qubit state tomography and post-measurement selection schemes, we
reconstruct the average output to characterize the fidelity with
which three different superconductor devices implement our chosen circuits.

In this report we present the results of these computations in order to
discuss the current state of development for publicly available superconducting
quantum computers. Our aim is to characterize the fidelity with which our
circuits can be implemented on the devices, and as such it is necessary to 

The purpose of this report is to examine various circuit in order to determine
if the post measurement method is valid for different circuits and if we can
improve the results form various devices by implementing 1- and 2 qubit state
tomography. In the case of for example quantum teleportation or entanglement
swapping circuit it is impossible to run the original circuits on real devices
since they require that initial measurements dictate which operators are used on
the qubits. This is impossible on IBMQ backends because they do not support
operations after measurements. The goal of these measurements is to determine
the fidelity of a prepared state to a target pure state. To this end we analyze
density matrices and Pauli set plots of the final states. Our state tomography
results also allow us to correct for readout error, which is the error caused by
the measurement of the qubits in different basis \cite{nielsen10_quant}.

In this report we first discuss the theory needed to understand state tomography
and readout error correction. Then we present our methods for developing Qiskit
software and explain the relevant circuits. In results chapter the measurement
data is shown and from this data we derive important performance metrics like
the fidelity for each circuit on different backends. Finally in the last chapter
we discuss the conclusion, recommendations and outlook.

%%% Local Variables:
%%% mode: latex
%%% TeX-master: "report"
%%% End:


\section{Theory}
In a classical computer its internal state is measured at different points in time in order to debug the system. However, for a quantum computer, the analogy would be the measurement of its density matrix, which is called state tomography. We first define the density matrix of a single qubit:
\begin{equation}
\rho=\frac{1}{2}\left(I+\sum_i\alpha_i\sigma_i\right)
\end{equation}

\section{Results}

\begin{frame}{Results}

\end{frame}

\section{Future Outlook}

%%% Local Variables:
%%% mode: latex
%%% TeX-master: "presentation"
%%% End:


\section{Conclusion}

Our research project was mainly focused on the fidelities of different circuits and if we were able to improve those for readout error using state tomography on various circuits. We conclude that we succeeded in recreating the final states using the post measurement method (except for the Grover circuit), since the simulator fidelities for the four circuits are approximately 1. The noisy simulator overestimated the device results for each circuit, and the error correction methods for one- and two-qubit systems improve the fidelity with respect to the device. The optimal backend for executing the relevant circuits is Yorktown, because it gives the best fidelity and the Pauli sets
%%% Local Variables:
%%% mode: latex
%%% TeX-master: "report"
%%% End:

%	REFERENCE LIST
\printbibliography

% ----------------------------------------------------------------------------------------
\end{document}