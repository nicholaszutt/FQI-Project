\section{Conclusion}

This research project characterized the performance of three IBMQ backends using
well-known circuits. We wanted to find the true performance of the circuits, and
to that end had to calibrate for readout error (which we have shown to be a
large source of error in the final, measured output of the superconducting
circuits). The circuit states were successfully reconstructed using one- and
two-qubit state tomography, employing the suitable post-measurement methods
where applicable (Teleportation and Entanglement Swapping).

As we ultimately found, the optimal backend for implementation was Yorktown.
Evidently, the advantages of greater inter-connectivity outweigh any
disadvantage from slightly lower gate fidelities (recall that Burlington,
despite higher single-qubit U$_2$ and CNOT fidelities frequently performed worse
than Yorktown). With the ability to create qubits with consistently high gate
fidelities, we suspect that the greatest improvement to devices will come in the
form of improving connectivity, rather than continuing to boost the performance
of gate operations. The backends with the highest ratio of connections to qubits
(Melbourne with 1.3$\bar{3}$ and Yorktown with 1.2 connections per qubit) often
outperform Burlington, a backend with just 0.8 connections per
qubit\footnote{Refer to \nameref{Specifications} for the diagrams.}. The number
of gates needed to implement a given circuit becomes very important, especially
as the circuits grow more complex.

As quantum computers continue to develop in the Noisy Intermediate-Scale Quantum
(NISQ) era, it is likely that different types of circuits will favour
implementation on qubits with very different physical properties. The various
approaches to qubits today, Nitrogen Vacancy centres in diamond, superconducting
LC oscillators, trapped-ions, etc, use different gates to construct a universal set,
which necessarily means a circuit that might be easy to implement in
superconductors would be much less suited to trapped-ions, or vice versa. An
interesting comparison to conduct in the future might be the performance of different
physical backends with regard to a handful of well-known protocols.

\newpage

%%% Local Variables:
%%% mode: latex
%%% TeX-master: "report"
%%% End: