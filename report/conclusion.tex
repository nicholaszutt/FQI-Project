\section{Conclusion}

Our research project was mainly focused on the fidelities of different circuits and if we were able to improve those for readout error using state tomography on various circuits. We conclude that recreation of the final states using the post measurement method (except for the Grover circuit) was a success, since the simulator fidelities for the four circuits are approximately 1. The noisy simulator overestimated the device results for each circuit, and the error correction methods for one- and two-qubit systems improve the fidelity with respect to the device. This means that we normally would overestimate the error originating from the device. The optimal backend for executing the relevant circuits is Yorktown, because it gives the best fidelity for the 3- and 4-qubit circuits. As mentioned in \nameref{Specifications}, the configuration of individual qubits in the device and in specific the number of connections. This is also why Burlington gives the worst fidelity of the backends, because it has the least amount of qubit connections. Our explanation for this effect is that the qubit network dictates how many real operations have to be done to run the circuits. So even though the device has a relatively low gate error rate, it could be the case that its accumulated error is higher, because it requires more gates to run the circuit due to the fact that it has a smaller amount of qubit connections.

{\color{red} \emph{You can add info on the individual circuits here...What went well and what didn't....Future research/prospects....}}
%%% Local Variables:
%%% mode: latex
%%% TeX-master: "report"
%%% End: