\section{Theory}
\subsection{One-Qubit State Tomography}
In a classical computer its internal state is measured at different points in time in order to debug the system. However, for a quantum computer, the analogy would be the measurement of its density matrix, which is called state tomography. Multiple copies of the state are required to distinguish between non-orthogonal states, which are generated by repeating the experiment. We first define the density matrix of a single qubit,
\begin{equation}
\rho=\frac{1}{2}\left(I+\sum_i\alpha_i\sigma_i\right)
\end{equation}
where $\sigma_i$ are all the Pauli-matrices and $\alpha_i$ are the real-valued coefficients. Using the trace orthogonality of the Pauli-matrices (because the trace of a operator gives the average value of the observable),
\begin{equation}
\Tr\left(\sigma_j\sigma_k\right)=2\delta_{jk}
\end{equation}
we can derive the real-valued coefficients by calculating the expectation values of the different Pauli-matrices.
\begin{equation}
\Tr\left(\rho\sigma_i\right)=\left\langle\sigma_i\right\rangle=\alpha_i
\end{equation}
By measuring the single qubit in the different basis ($X$,$Y$ and $Z$) we can derive these expectation values. This requires a repeated preparation and measuring of the final state, and in the limit of a large sample size this will give a good estimation of $\rho$. In reality, the measured expectation values are estimations of $\left\langle \hat{X}\right\rangle$, $\left\langle \hat{Y}\right\rangle$, $\left\langle \hat{Z}\right\rangle$. Often in a quantum computer the measurements are only done in the $Z$-basis. Other operators are realized using rotation operators before the final measurement.

In order to convert the estimated- to real expectation values we correct for readout error, which will give us a better estimation of the density matrix $\rho$. If $\epsilon_{10},\epsilon_{01}$ are the probabilities that a $\left|0\right\rangle$ state gives an eigenvalue back of -1 and a $\left|1\right\rangle$ state which is measured as a eigenvalue 1 (note that this is not a quantum effect but classical error), and if $\alpha,\beta$ are coefficients of the final state $\left|\psi\right\rangle=\alpha\left|0\right\rangle+\beta\left|1\right\rangle$, then the measured expectation value $\overline{m}$ in the $Z$-basis is
\begin{equation*}
\begin{split}
\overline{m} =\left(1-\epsilon_{10}\right)\left|\alpha\right|^2+\epsilon_{01}\left|\beta\right|^2-\left(1-\epsilon_{01}\right)\left|\beta\right|^2\\
-\epsilon_{10}\left|\alpha\right|^2\\=\left(\epsilon_{01}-\epsilon_{10}\right)+\left(1-\epsilon_{01}-\epsilon_{10}\right)\left(\left|\alpha\right|^2-\left|\beta\right|^2\right)\\
\end{split}
\end{equation*}
\begin{equation}
\begin{split}
=\beta_0+\beta_1\left\langle \hat{Z}\right\rangle
\end{split}
\end{equation}
From error measurements $\beta_0$ and $\beta_1$ are derived by measuring different states on all relevant qubits. The assumptions we make here are the following: initialization of the initial me create for the readout error measurements are almost perfect (99\% accurate) and the gates used to create the final states also have a high fidelity (exceeding 99.9\%).
From this expression $\left\langle \hat{Z}\right\rangle$ and the other expectation values $\left\langle \hat{\sigma}_i\right\rangle$ are derived from all the different measurements $\overline{m}_i$.
\begin{equation}
\begin{split}
\left\langle \hat{\sigma}_i\right\rangle=\frac{\overline{m}_i-\beta_0}{\beta_1}
\end{split}
\end{equation}
So the final density matrix $\rho_{fin}$ is defined as:
\begin{equation}
\begin{split}
\rho_{fin}=\frac{1}{2}\left(I+\sum_i\frac{\overline{m}_i-\beta_0}{\beta_1}\sigma_i\right)
\end{split}
\end{equation}
\subsection{Two-Qubit State Tomography}
This procedure can be generalized to more qubits, so similar to the one-qubit case the density matrix on $n$ number of qubits is expressed as
\begin{equation}
\rho=\sum_{\vec{v}_i}\frac{\Tr\left(\sigma_{v_1}\otimes\cdots\otimes\sigma_{v_n}\rho\right)\sigma_{v_1}\otimes\cdots\otimes\sigma_{v_n}}{2^n}
\end{equation} 
where we sum over vectors $\vec{v}_i=\left(v_1,...,v_n\right)$. The entries can be all the different combinations of basis. What these products of Pauli matrices mean is that we measure different bits in various basis. Rewriting this general expression for a two qubit system gives
\begin{equation*}
\rho=\frac{1}{4}\sum_{\vec{v}_i}\Tr\left(\sigma_{v_1}\otimes\sigma_{v_2}\rho\right)\sigma_{v_1}\otimes\sigma_{v_2}
\end{equation*}
\begin{equation}
=\frac{1}{4}\sum_{\vec{v}_i}\left\langle\sigma_{v_1}\otimes\sigma_{v_2}\right\rangle\sigma_{v_1}\otimes\sigma_{v_2}
\end{equation}
where we sum over all the combinations of basis ($I$,$X$,$Y$ and $Z$). Note that for a two qubit system we can choose to only measure one bit, that why the $I$ matrix is included here it means you do not measure that particular qubit. So in total we have fifteen coefficient which can be extracted from 9 different circuits for the basis combinations. The nine measurement basis are the two-qubit correlations and then we have 6 basis containing the identity (not counting $II$). These six are the Bloch vectors of the most- and least significant bits, and those are derived from the original measurements by counting up all counts of the basis where there is now an identity matrix. For example if we want to know the expectation value of $\left\langle XI\right\rangle$ we can use three different measurement results ($\left\langle XX\right\rangle$, $\left\langle XY\right\rangle$ and $\left\langle XZ\right\rangle$), and in this case combine the counts from the least significant qubit.

Again we can correct for readout error, but for two-qubit systems this is a bit more complicated. Ideally we want the measured expectation value in a particular basis for two different qubits to coincide with the actual expectation values of the most-, least significant qubit and the two-qubit correlation. But the readout is compromised by readout error and crosstalk between the qubits. As a result of these errors the MSQ, LSQ and correlations results are changed.
\begin{equation}
\begin{pmatrix}
\overline{m}_{MSQ}  \\
\overline{m}_{LSQ}  \\
\overline{m}_{corr}
\end{pmatrix}=\begin{pmatrix}
\beta_{0,M}  \\
\beta_{0,L}  \\
\beta_{0,corr}
\end{pmatrix}
\end{equation}












 