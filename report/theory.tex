\section{Theory}

\subsection{One-Qubit State Tomography} In a classical computer the internal
state is measured at different points in time in order to read data, debug the
system, or do computation. Since measurement in quantum mechanics destroys the
state being measured, though, we cannot simply measure a quantum state once and
learn everything there is to know about it. In order to characterize quantum
states in computers, one must resort to state tomography, the process of
measuring an ensemble of states in many different bases to reconstruct the
initial state's density matrix and thus find out what state was realized
\textit{before} the measurement, rather than simply finding what the state has
collapsed to \textit{after} the measurement.
We first define the density matrix of a single
qubit,

\begin{equation} \rho=\frac{1}{2}\left(\mathbb{1}+\sum_{i=1}^3\alpha_i\sigma_i\right)
\end{equation}

where $\mathbb{1}$ is the 2-D identity matrix, $\sigma_i$ are all the Pauli-matrices and $\alpha_i$ are
real-valued coefficients \cite{nielsen10_quant}. Using the trace orthogonality of the
Pauli-matrices (because the trace of an operator gives the average value of the
observable),

\begin{equation} \Tr\left(\sigma_j\sigma_k\right)=2\delta_{jk}
\end{equation}

we can derive the real-valued coefficients by calculating the
expectation values of the different Pauli-matrices.

\begin{equation}
\Tr\left(\rho\sigma_i\right)=\left\langle\sigma_i\right\rangle=\alpha_i
\end{equation}

By measuring the single qubit in the different bases ($X$, $Y$ and
$Z$) we can derive these expectation values. This requires a repeated
preparation and measuring of the final state, and in the limit of a large sample
size this reconstructed density matrix approaches the state's density matrix $\rho_s$. In reality, the measured
expectation values are estimations of $\langle \hat{X}\rangle$,
$\langle \hat{Y}\rangle$, $\left\langle \hat{Z}\right\rangle$. Often,
as in this case,
in a quantum computer the measurements are only done in the $Z$-basis. Other
operators are realized using rotation operators before the final measurement.

We wish to characterize the fidelity of the circuits without including the error
that results from measurements. Therefore, in order to convert these expectation values we correct for
readout error, which will give us a better estimation of the density matrix
$\rho_s$. If $\epsilon_{10}$, $\epsilon_{01}$ are the probabilities that a
$|0\rangle$ state returns an eigenvalue of -1 and a
$|1\rangle$ state returns an eigenvalue of 1 respectively (note that this
is not a quantum effect but classical error), and if $\alpha$, $\beta$ are
coefficients of the final state
$|\psi\rangle=\alpha|0\rangle+\beta|1\rangle$,
then the measured expectation value $\overline{m}$ in the $Z$-basis is

\begin{align*}
  \overline{m} &=\left(1-\epsilon_{10}\right)\left|\alpha\right|^2+\epsilon_{01}\left|\beta\right|^2\\
&-\left(1-\epsilon_{01}\right)\left|\beta\right|^2
-\epsilon_{10}\left|\alpha\right|^2\\
&=\left(\epsilon_{01}-\epsilon_{10}\right)+\left(1-\epsilon_{01}-\epsilon_{10}\right)\left(\left|\alpha\right|^2-\left|\beta\right|^2\right)\\
\end{align*}

Using the last line we can define the parameters $\beta_0$ and $\beta_1$,

\begin{equation}
  \begin{split}
   \overline{m}_Z =\beta_0+\beta_1\left\langle \hat{Z}\right\rangle
\end{split}
\end{equation}

From error measurements $\beta_0$ and $\beta_1$ are derived by
measuring different states on all relevant qubits. These estimates of $\beta_0$
and $\beta_1$ are accurate because we know that the initialization of the states created for the readout error
measurements is almost perfect (99\% accurate) and the gates used to create the
final state measurements (in the case of $\langle \hat{X}\rangle$,
$\langle \hat{Y}\rangle$) also have an extremely high fidelity (exceeding
99.9\%)\cite{ibmq_burlington,ibmq_16_melbourne,ibmq_yorktown}. From this
expression $\langle \hat{Z}\rangle$ and the other expectation values
$\langle \hat{\sigma}_i\rangle$ are derived from all the different
measurements $\overline{m}_i$.

\begin{equation}
\begin{split} \left\langle
\hat{\sigma}_i\right\rangle=\frac{\overline{m}_i-\beta_0}{\beta_1}
\end{split}
\end{equation} So the final density matrix $\rho_{fin}$ is defined as:

\begin{equation}
\begin{split}
\rho_{fin}=\frac{1}{2}\left(I+\sum_{i=1}^3\frac{\overline{m}_i-\beta_0}{\beta_1}\sigma_i\right)
\end{split}
\end{equation}


\subsection{Two-Qubit State Tomography} This procedure can be generalized to
more qubits, and to do so we start with the $n$-qubit generalization of the
density matrix 

\begin{equation}
\rho=\sum_{\vec{v}_i}\frac{\Tr\left(\sigma_{v_1}\otimes\cdots\otimes\sigma_{v_n}\rho\right)\sigma_{v_1}\otimes\cdots\otimes\sigma_{v_n}}{2^n}
\end{equation}

where we sum over vectors $\vec{v}_i=\left(v_1,...,v_n\right)$ \cite{nielsen10_quant}.
The entries can be all the different combinations of basis. What these products
of Pauli matrices mean is that we measure different bits in various basis.
Rewriting this general expression for a two qubit system gives

\begin{equation*}
\rho=\frac{1}{4}\sum_{\vec{v}_i}\Tr\left(\sigma_{v_1}\otimes\sigma_{v_2}\rho\right)\sigma_{v_1}\otimes\sigma_{v_2}
\end{equation*}

\begin{equation}
=\frac{1}{4}\sum_{\vec{v}_i}\left\langle\sigma_{v_1}\otimes\sigma_{v_2}\right\rangle\sigma_{v_1}\otimes\sigma_{v_2}
\end{equation}

where we sum over all the combinations of basis ($\mathbb{1}$, $X$, $Y$ and
$Z$). So in total we have fifteen coefficients which can be extracted
from 9 different circuits for the basis combinations. The nine measurement bases
are the two-qubit correlations and then we have 6 expectation values containing the identity
(we do not count $\mathbb{1} \otimes \mathbb{1}$ as it contributes 1 to the sum always). These six are the Bloch vectors of the most- and least
significant bits, and those are derived from the original measurements by
counting up all counts of the basis where there is now an identity matrix. For
example if we want to know the expectation value of $\langle
XI\rangle$ we can use three different measurement results ($\langle
XX\rangle$, $\langle XY\rangle$ and $\langle
XZ\rangle$), and in this case combine the counts from the least
significant qubit.

Again we can correct for readout error, but for two-qubit systems this is a bit
more complicated. Ideally we want the measured expectation value in a particular
basis for two different qubits to coincide with the actual expectation values of
the most-, least significant qubit and the two-qubit correlation. But the
readout is compromised by readout error and crosstalk between the qubits. As a
result of these errors the most-significant qubit (MSQ), least-significant qubit
(LSQ) and correlations results are all affected. We
define the measured result in a specific basis $\overline{m}_{ij}$ as

\begin{equation}
\label{beta}
\begin{split}
\begin{pmatrix} \overline{m}_{ij,MSQ} \\ \overline{m}_{ij,LSQ} \\
\overline{m}_{ij,corr}
\end{pmatrix}=\begin{pmatrix} \beta_{0,M} \\ \beta_{0,L} \\ \beta_{0,corr}
\end{pmatrix}+\\\begin{pmatrix} \beta_{1,M}&\beta_{2,M}&\beta_{3,M} \\
\beta_{1,L}&\beta_{2,L}&\beta_{3,L} \\
\beta_{1,corr}&\beta_{2,corr}&\beta_{3,corr}
\end{pmatrix}\begin{pmatrix} \left\langle I\sigma_j\right\rangle \\ \left\langle
\sigma_iI\right\rangle \\ \left\langle \sigma_i\sigma_j\right\rangle
\end{pmatrix}
\end{split}
\end{equation}

Note that instead of 2 beta's we now have 4 for the offset
$\beta_0$ and for the MSQ, LSQ and correlations correction for each type of
measurement (in the following order: $\beta_1$, $\beta_2$ and $\beta_3$). If we
define $B$ as the matrix containing beta's (except for $\beta_0$) then we can
rewrite the expectation values in terms of the measured results as follows:
\begin{equation}
\begin{split}
\begin{pmatrix} \left\langle I\sigma_j\right\rangle_{est} \\ \left\langle
\sigma_iI\right\rangle_{est} \\ \left\langle \sigma_i\sigma_j\right\rangle_{est}
\end{pmatrix}=B^{-1}\left(\begin{pmatrix} \overline{m}_{ij,MSQ} \\
\overline{m}_{ij,LSQ} \\ \overline{m}_{ij,corr}
\end{pmatrix}-\begin{pmatrix} \beta_{0,M} \\ \beta_{0,L} \\ \beta_{0,corr}
\end{pmatrix}\right)
\end{split}
\end{equation} We need four calibration circuits to calibrate all the $\beta$
coefficients: one circuit that has no operators, a circuit with an X-gate for
the MSQ, one for the LSQ and one circuit where both qubits have a X-gate. From
these configurations we calculate the desired parameters, where the circuits
have letters A,B,C and D in that specific order.
\begin{equation}
\begin{pmatrix} \overline{m}_{A,k} \\ \overline{m}_{B,k} \\ \overline{m}_{C,k}
\\ \overline{m}_{D,k}
\end{pmatrix}=\begin{pmatrix} 1&1&1&1\\ 1&1&-1&-1\\ 1&-1&1&-1\\ 1&-1&-1&1
\end{pmatrix}\begin{pmatrix} \beta_{0,k} \\ \beta_{1,k} \\ \beta_{2,k} \\
\beta_{3,k}
\end{pmatrix}
\end{equation} where $k$ can be MSQ, LSQ or correlated and the matrix is called
$M_k$. These relation are derived from formula \ref{beta}, where the expectation
values are either 1 or -1 depending on the type of circuit. Finally we express
the beta's in terms of these calibration results:
\begin{equation}
\begin{pmatrix} \beta_{0,k} \\ \beta_{1,k} \\ \beta_{2,k} \\ \beta_{3,k}
\end{pmatrix}=M_k^{-1}
\begin{pmatrix} \overline{m}_{A,k} \\ \overline{m}_{B,k} \\ \overline{m}_{C,k}
\\ \overline{m}_{D,k}
\end{pmatrix}
\end{equation} In conclusion, it is also possible to correct for readout error
in a two-qubit system using calibration circuits and then transforming the
results with matrices consisting of the derived beta's.

%%% Local Variables:
%%% mode: latex
%%% TeX-master: "report"
%%% End: