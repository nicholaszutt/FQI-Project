\section{Theory}
\subsection{One-Qubit State Tomography}
In a classical computer its internal state is measured at different points in time in order to debug the system. However, for a quantum computer, the analogy would be the measurement of its density matrix, which is called state tomography. We first define the density matrix of a single qubit,
\begin{equation}
\rho=\frac{1}{2}\left(I+\sum_i\alpha_i\sigma_i\right)
\end{equation}
where $\sigma_i$ are all the Pauli-matrices and $\alpha_i$ are the real-valued coefficients. Using the trace orthogonality of the Pauli-matrices,
\begin{equation}
\Tr\left(\sigma_j\sigma_k\right)=2\delta_{jk}
\end{equation}
we can derive the real-valued coefficients by calculating the expectation values of the different Pauli-matrices.
\begin{equation}
\Tr\left(\rho\sigma_i\right)=\left\langle\sigma_i\right\rangle=\alpha_i
\end{equation}
By measuring the single qubit in the different basis ($X$,$Y$ and $Z$) we can derive these expectation values. This requires a repeated preparation and measuring of the final state. In reality, the measured expectation values are estimations of $\left\langle \hat{X}\right\rangle$, $\left\langle \hat{Y}\right\rangle$, $\left\langle \hat{Z}\right\rangle$. Often in a quantum computer the measurements are only done in the $Z$-basis. Other operators are realized using rotation operators before the final measurement.

In order to convert the estimated- to real expectation values we correct for readout error, which will give us a better estimation of the density matrix $\rho$. If $\epsilon_{10},\epsilon_{01}$ are the probabilities that a $\left|0\right\rangle$ state gives an eigenvalue back of -1 and a $\left|1\right\rangle$ state which is measured as a eigenvalue 1 (note that this is not a quantum effect but classical error), and if $\alpha,\beta$ are coefficients of the final state $\left|\psi\right\rangle=\alpha\left|0\right\rangle+\beta\left|1\right\rangle$, then the measured expectation value $\overline{m}$ in the $Z$-basis is
\begin{equation*}
\begin{split}
\overline{m} =\left(1-\epsilon_{10}\right)\left|\alpha\right|^2+\epsilon_{01}\left|\beta\right|^2-\left(1-\epsilon_{01}\right)\left|\beta\right|^2\\
-\epsilon_{10}\left|\alpha\right|^2\\=\left(\epsilon_{01}-\epsilon_{10}\right)+\left(1-\epsilon_{01}-\epsilon_{10}\right)\left(\left|\alpha\right|^2-\left|\beta\right|^2\right)\\
\end{split}
\end{equation*}
\begin{equation}
\begin{split}
=\beta_0+\beta_1\left\langle \hat{Z}\right\rangle
\end{split}
\end{equation}
From error measurements $\beta_0$ and $\beta_1$ are derived by measuring different states on all relevant qubits. The assumptions we make here are the following: initialization of the initial me create for the readout error measurements are almost perfect (99\% accurate) and the gates used to create the final states also have a high fidelity (exceeding 99.9\%).
From this expression $\left\langle \hat{Z}\right\rangle$ and the other expectation values $\left\langle \hat{\sigma}_i\right\rangle$ are derived from all the different measurements $\overline{m}_i$.
\begin{equation}
\begin{split}
\left\langle \hat{\sigma}_i\right\rangle=\frac{\overline{m}_i-\beta_0}{\beta_1}
\end{split}
\end{equation}
So the final density matrix $\rho_{fin}$ is defined as,
\begin{equation}
\begin{split}
\rho_{fin}=\frac{1}{2}\left(I+\sum_i\frac{\overline{m}_i-\beta_0}{\beta_1}\sigma_i\right)
\end{split}
\end{equation}
\newpage
\subsection{Two-Qubit State Tomography}

  