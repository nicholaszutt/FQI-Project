\section{Introduction}

\lettrine[nindent=0em,lines=3]{Q} uantum computers exploit quantum mechanical
phenomena in order to perform calculations and manipulate data in ways that
would be impossible on classical computers. Quantum algorithms have been
developed to teleport data from one place to another, search databases
efficiently, and factor large numbers quickly, to name a few possible applications. These algorithms will comprise
the building blocks of the quantum computers of the future, and the current
efforts towards realizing a truly universal quantum computer centre around
improving the manipulation of quantum bits (or qubits), the basic units of
computation in these algorithms.

There are many approaches to creating these qubits. Useful qubits instantiate a
set of properties that can be at odds with each other at times. For example, a
desirable trait like the ability to accurately control the state of the qubit
often conflicts with the desire for the qubit to be long-lived (i.e. to have a
long coherence time where its quantum state is safe from environmental
degradation).

The various implementations make the trade-off between desirable traits in
different ways, and a leading approach for creating qubits uses LC circuits in
superconductors \cite{kjaergaard19_super_qubit}. Already in 2014, the first
demonstrated universal gate set on superconducting qubits with an average gate
fidelity over 99 per cent for all gates was realized
\cite{barends14_super_quant_circuit_at_surfac}. As the field has progressed
since then, superconducting qubits have only become more attractive as the
building blocks for quantum computers. Now, multi-qubit devices that use
superconductors are available publicly, and capable of performing quantum
computations that implement small (less than 16 qubits, with limited depth)
circuits.

We have simulated and executed a handful of foundational circuits on
superconducting devices provided publicly by the IBM Quantum Experience. We use
Qiskit, an open-source quantum computing software development
platform which has quickly become the most popular means with which to program
circuits on publicly available quantum computers. Using
1- and 2-qubit state tomography and post-measurement selection schemes, we
reconstruct the average output to characterize the fidelity with
which three different superconductor devices implement our chosen circuits.

In this report we present the results of these computations in order to
discuss the current state of development for publicly available superconducting
quantum computers. Our aim is to characterize the fidelity with which our
circuits can be implemented on the devices, and as such it is necessary to 

The purpose of this report is to examine various circuit in order to determine
if the post measurement method is valid for different circuits and if we can
improve the results form various devices by implementing 1- and 2 qubit state
tomography. In the case of for example quantum teleportation or entanglement
swapping circuit it is impossible to run the original circuits on real devices
since they require that initial measurements dictate which operators are used on
the qubits. This is impossible on IBMQ backends because they do not support
operations after measurements. The goal of these measurements is to determine
the fidelity of a prepared state to a target pure state. To this end we analyze
density matrices and Pauli set plots of the final states. Our state tomography
results also allow us to correct for readout error, which is the error caused by
the measurement of the qubits in different basis \cite{nielsen10_quant}.

In this report we first discuss the theory needed to understand state tomography
and readout error correction. Then we present our methods for developing Qiskit
software and explain the relevant circuits. In results chapter the measurement
data is shown and from this data we derive important performance metrics like
the fidelity for each circuit on different backends. Finally in the last chapter
we discuss the conclusion, recommendations and outlook.

%%% Local Variables:
%%% mode: latex
%%% TeX-master: "report"
%%% End:
